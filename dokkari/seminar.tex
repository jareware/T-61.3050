% For easier proof-reading, use the single-column, double-spaced layout:
\documentclass{netsec2012}

% Final Paper use double-column, normal line spacing. Comment the
% above line and uncomment the following line when you are writing
% Full paper and Final paper!  
%\documentclass[cameraready]{netsec2012}

\begin{document}

%=========================================================

\title{Single character OCR using Support Vector Machine}

\author{Olli Jarva \& Jarno Rantanen \\
        Aalto University School of Science \\
	\texttt{olli@jarva.fi} \texttt{\&} \texttt{jarno@jrw.fi}}
\maketitle

%==========================================================

\begin{abstract}

This paper describes a solution to an optical character recognition problem for bitmap characters
using Support Vector Machines with an RBF kernel, including a description of RBF parameter search
and bitmap normalization.  Classification performance of 90.8\% was achieved against a given
training set of 40000 correctly labelled samples \cite{training_set}.

\vspace{3mm}
\noindent KEYWORDS: SVM, Support Vector Machine, RBF, OCR, Character Recognition

\end{abstract}

%============================================================




% Remember that one of the objectives of the course is to teach you how to apply machine learning methods in a principled way, instead of just running some black box method on data and hoping that it will perform well. Therefore in the report it is not enough, e.g., just to explain the algorithm you implemented. Typically, the report should also include a discussion of why you selected a particular method (pros and cons), the principles of the method, validation of your approach (e.g., feature selection, model complexity selection, validation, estimation of generalization error etc.), your conclusions and other relevant issues.
% The factual content of the report must be correct and relevant. There should be sufficient amount of essential information (the length of the answer is not merit in itself). The report should give an impression that you understand what you are doing and therefore you are able to apply what you have learned in the course to machine learning problems.
% You must substantiate all of your claims. For example, you can't claim that your method will generalize well to new data if you haven't shown this, e.g., experimentally, by logical (mathematical) argumentation or by appropopriate citation. You must make a clear distinction between facts, substantiated claims and opinions (opinions being unsubstantiated claims).



\section{Dataset description}

Training dataset:

\begin{itemize}
\item n=42152
\item 16 x 8 black and white bitmaps
\item Lowercase characters, n=26 (a-z)
\end{itemize}

Testing dataset:

\begin{itemize}
\item n=10000
\item Same format
\end{itemize}

\section{Method selection/Why SVM?}

\begin{itemize}
\item What other options were there
\item Why we chose SVM?
\item A nod to why we think we chose correctly
\end{itemize}

\section{What is SVM?}

\begin{itemize}
\item http://www.csie.ntu.edu.tw/~cjlin/papers/guide/guide.pdf
\item http://www.ivanciuc.org/Files/Reprint/Ivanciuc\_SVM\_CCR\_2007\_23\_291.pdf
\item RBF kernel: $K(x_i, x_j) = exp(-\gamma || x_i - x_j ||^2), \gamma > 0$
\end{itemize}

Optimization problem $(x_i, y_i), i = 1, ..., l$ where $x_i$ is ... (\cite{libsvm_guide}):

minimize $w,b,\xi$: $\frac{1}{2}w^Tw + C \sum_{i=1}^l\xi_i$

subject to $y_i(w^T \phi(x_i) + b) \ge 1 - \xi_i, \xi_i \ge 0$

\section{Character preprocessing}

\begin{itemize}
\item Minimize noise by moving characters to bottom left corner. 0.5\% improvement
\end{itemize}

\section{RBF kernel parameter search}

$\gamma$ and $C$

\begin{itemize}
\item Initial search space 2**x for x in range(-15, 15)
\item Select best area for next round
\item Validate by taking final arguments and calculating error rates for +- few percent for both variables.
\end{itemize}

\section{Results and performance}

\begin{itemize}
\item k-fold cross validation: k=20, error rate 11\%
\item k-fold cross validation: k=5, error rate 11.5\%
\item One iteration with training set n=40000 and validation set n=2152 about 17 min with 2.1GHz Xeon (single thread)
\item about 300MB of memory for training set n=42152
\item Predicting one character: about 2 milliseconds

\end{itemize}

\section{Quick comparison to other algorithms}

\begin{itemize}
\item kNN (+PCA/LDA)
\item ...?
\end{itemize}


\cite{albanese12mlpy}

%============================================================

\bibliography{references}
\end{document}

